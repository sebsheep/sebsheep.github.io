\documentclass[10pt,a4paper]{moderncv}
\moderncvtheme[blue]{classic}               
\usepackage[utf8]{inputenc}
\usepackage[scale=0.85]{geometry}
\usepackage[frenchb]{babel}
\usepackage{tikz}
\usepackage{url}
\usepackage{multicol}
\usetikzlibrary{calc}
\usetikzlibrary{shapes,snakes}
\definecolor{color1}{rgb}{0.22,0.45,0.70}% light blue
\newcommand{\ep}{\tikz{\node[draw=color1,fill=color1,star point ratio=2.5,
,minimum size=2mm,inner sep = 0pt,star,star points=5] {};}}
\newcommand{\ec}{\tikz{\node[draw=color1,minimum size=2mm,star point ratio=2.5, inner sep = 0pt,star,star points=5] {};}}
 
\firstname{\textrm{Sébastien}}
\familyname{\textsc{Besnier}}
\title{Développeur matheux}   
\address{92 avenue du général Leclerc}{91120 Palaiseau}    
\mobile{06 31 72 16 79}                    
\email{sebastien.fabrice.besnier@gmail.com}
\homepage{besnier.se}                
\begin{document}
\vspace*{-2.5em}
\maketitle


Je suis intéressé par le développement d'applications incluant des défis techniques et/ou technologiques. Pour une description plus détaillée de mon
parcours, vous pouvez consulter ma page: \url{http://besnier.se}.

\section{Expériences}


\cventry{2015--}{Thèse en cryptogrpahie}{CEA-LIST}{91120 Palaiseau}{}{Mon sujet traite des Physically Unclonable Functions. L'un des principaux objectifs est de construire un algorithme de chiffrement utilisant de façon intrinsèque la variabilité des composants.}

\cventry{2016}{Cours de conception d'applications web}{ISTY}{Université de Versailles}{}{J'ai encadré les étudiants de master 1 en informatique durant les séances de TP/TD du cours Application Web et Sécurité. Voir le contenu du cours:
\url{http://defeo.lu/aws}.}

\cventry{2012--2015}{Enseignant de mathématiques et informatique}{Académie de Versailles}{}{}{}

\cventry{2011--}{Directeur de colonies de vacances}{Vitacolo}{}{}{Direction de 5 colonies de vacances: recrutement du personnel, gestion de planning, tenue des comptes, ...}

\section{Formation}
\cventry{2013--2014}{Master 2 <<Algèbre Appliquée à la Cryptographie
     et au Calcul Formel>>} {Faculté de Versailles}{}{}{
     Stage: développement d'un framework pour les isogénies entre
     courbes elliptiques en Sage\\ Rang: 2{$^e$}/10.} 

 \cventry{2011--2012}{Agrégation externe de
     mathématiques}{option informatique}{}{}{Reçu 85{$^e$}/ 308.}




\section{Projets}

\cventry{2015--2016 \emph{(en cours)}}{Outil de gestion de personnels
     et d'événements}{pour Vitacolo}{}{} {Site internet ouvert à tous
     permettant: le recrutement des animateurs et directeurs, la
     coordination de plusieurs séjours et l'élaboration de projets. 
     Technologie utilisée:
      Django/Python.}


\cventry{2014--2015 \emph{}}{Compilation multi-cible}{}{}{}{Écriture d'un générateur de code
     en Julia produisant des fonctions cryptographiques en C et en
     Java à partir d'une description haut niveau de ces fonctions. Voir 
     \url{http://github.com/sebsheep/bbox-julia}.}

\cventry{2015--}{Assistant de preuves interactif}{}{}{}{Le but de ce projet est d'illustrer aux enfants qu'une preuve mathématique est essentiellement un procédé ''mécanique''. Voir les sources \url{http://github.com/sebsheep/poussin} ou une démo: \url{http://poussin-interactif.herokuapp.com/}.}



\section{Compétences}


\begin{minipage}{0.35\linewidth}
  \renewcommand{\arraystretch}{1.5}
  \begin{tabular}{r@{\quad}l@{\quad~}r@{\quad}l}
    Python&\ep\ep\ep\ep\ep  &Node.js &\ep\ep\ec\ec\ec\\
    C/C++&\ep\ep\ep\ec\ec &Haskell&\ep\ep\ep\ec\ec\\
    Django& \ep\ep\ep\ep\ec & JavaScript &\ep\ep\ec\ec\ec\\
  \end{tabular}

\end{minipage}
\begin{minipage}[h!]{0.65\linewidth}
  \begin{itemize}
      \item[\color{color1}$\bullet$] Maîtrise des notions ``théoriques'' de
    l'informatique: complexité, automates/expressions régulières,
    automates à piles/grammaire, théorie des graphes.
      \item[\color{color1}$\bullet$] Connaissances sur la conception de bases de
    données.
      \item[\color{color1}$\bullet$] Administration système Linux: notions en bash .
  \end{itemize}

\end{minipage}







   

% \cventry{2009--2010}{Licence et Magistère de Mathématiques
% Fondamentales}{Faculté d'Orsay}{}{}{Mention bien, un module
% d'informatique théorique. Stage d'un mois dans le laboratoire LLR de
% l'école Polytechnique (Palaiseau): implémentation d'un filtre de
% Kalman.}  \cventry{2008--2009}{DEUG de Mathématiques}{Faculté
% d'Orsay}{}{}{Mention très bien.}

% \subsection{Productions écrites}
% \vspace{-0.4cm}
% \begin{center}
%   \begin{minipage}{0.7\linewidth}
%     Les documents présentés ici sont principalement des mémoires
%     réalisés durant mes études à l'université d'Orsay. Ils sont
%     décrits en détail et accessibles à l'adresse:
%     http://sebastien.besnier1.free.fr/
%   \end{minipage}
% \end{center}
% \cventry{2012}{Fonctions Elliptiques}{encadré par O. Bouillot}{}{}{}
% \cventry{2011}{Calcul Variationnel}{encadré par
% F. Santambrogio}{}{}{} \cventry{2011}{Algorithmique
% Quantique}{encadré par S. Laplante}{}{}{} \cventry{2009}{Articles de
% vulgarisation mathématique et informatique}{}{}{}{} \newpage




\section{Autre: musique classique}
\cventry{Guitare}{\'Etude en conservatoire}{de 1994 à 2006}{niveau de
     << supérieur >>}{}{}
\cventry{Piano}{\'Etude en autodidacte}{de 2000 à 2004}{}{}{}
\cventry{Chant lyrique}{\'Etude en conservatoire}{de 2004 à
     2006}{niveau de << deuxième cycle >>}{}{} 
      \cventry{Violoncelle}{\'Etude en
     conservatoire}{à partir de 2012}{}{}{}

\end{document}