\documentclass[10pt,a4paper]{moderncv}
\moderncvtheme[blue]{classic}               
\usepackage[utf8]{inputenc}
\usepackage[scale=0.85]{geometry}
\usepackage[english]{babel}
\usepackage{tikz}
\usepackage{url}
\usepackage{multicol}
\usetikzlibrary{calc}
\usetikzlibrary{shapes,snakes}
\definecolor{color1}{rgb}{0.22,0.45,0.70}% light blue
\newcommand{\ep}{\tikz{\node[draw=color1,fill=color1,star point ratio=2.5,
,minimum size=2mm,inner sep = 0pt,star,star points=5] {};}}
\newcommand{\ec}{\tikz{\node[draw=color1,minimum size=2mm,star point ratio=2.5, inner sep = 0pt,star,star points=5] {};}}

\firstname{\textrm{Sébastien}}
\familyname{\textsc{Besnier}}
\title{Computer developer with maths skills}   
\address{92 avenue du général Leclerc}{91120 Palaiseau, France}       
\mobile{06 31 72 16 79}                    
\email{sebastien.fabrice.besnier@gmail.com}
\homepage{besnier.se}                
\begin{document}
\vspace*{-2.5em}
\maketitle




I am interested in code development, especially when there's technically challenging aspects to it.  For a more detailed version of my resume, see my page: \url{http://besnier.se/en}.

\section{Expériences}


\cventry{2015--}{PhD in cryptography}{CEA-LIST}{91120 Palaiseau (France)}{}{My subject is about Physically Unclonable Functions: the main interest is to find a way to build cipher algorithm using the intrinsic variability of the components.}

\cventry{2016}{Tutor in web application conception class}{ISTY}{Université de Versailles (France)}{}{I've taught during practical sessions of the class ''Application Web et Sécurité'' (''Web application and security'') in first year of Computer Science master. See the content of the class:
\url{http://defeo.lu/aws}.}

\cventry{2012--2015}{Math and computer science teahcer }{Académie de Versailles (France)}{}{}{}

\cventry{2011--}{Child's summer camp direction }{Vitacolo}{}{}{I have been head of 5 children summer camps: staff recruitment, planning, accounting, ...}

\section{Education}
\cventry{2013--2014}{Master 2 <<Algèbre Appliquée à la Cryptographie
     et au Calcul Formel>> ("Algebra applied to cryptography and formal calculation") } {Faculté de Versailles (France)}{}{}{
     Internship: improvement of the isogenies support in Sage.\\ Rank: 2/10.} 

 \cventry{2011--2012}{Agrégation externe de
     mathématiques}{Computer Science option}{}{}{The \textbf{agrégation de mathématiques} is a selective exam in mathematics allowing to teach in the french system up to about the end of the bachelor's degree (the ''licence'' in the European LMD system) in mathematics.
\\Rank! 85/ 308.}




\section{Programs}

\cventry{2015--2016 \emph{(in progress)}}{Help desk for summer camp organization}{for Vitacolo}{}{} {Technologies used: Python, Django, twitter-bootstrap, a bit of jquery.}


\cventry{2014--2015 \emph{}}{Multi-target compilation}{}{}{}{This project aims to produce (cryptographic) code in several languages (namely C, Java and Python) which behaves exactly in the same manner.See 
     \url{http://github.com/sebsheep/bbox-julia}.}

\cventry{2015--}{Interactive proof assistant}{}{}{}{The goal of this project is to illustrate to children that a mathematical proof is essentially "mechanic". See the source code \url{http://github.com/sebsheep/poussin} or a live demo: \url{http://poussin-interactif.herokuapp.com/}.}



\section{Competences}


\begin{minipage}{0.35\linewidth}
  \renewcommand{\arraystretch}{1.5}
  \begin{tabular}{r@{\quad}l@{\quad~}r@{\quad}l}
    Python&\ep\ep\ep\ep\ep  &Node.js &\ep\ep\ec\ec\ec\\
    C/C++&\ep\ep\ep\ec\ec &Haskell&\ep\ep\ep\ec\ec\\
    Django& \ep\ep\ep\ep\ec & JavaScript &\ep\ep\ec\ec\ec\\
  \end{tabular}

\end{minipage}
\begin{minipage}[h!]{0.65\linewidth}
  \begin{itemize}
      \item[\color{color1}$\bullet$] Masters theoretical notions in Computer Sciences:
      complexity, finite state machine/regular expressions, pushdown automaton/grammars,
      graph theory.
       
      \item[\color{color1}$\bullet$] Knowledges on database design.
      \item[\color{color1}$\bullet$] Linux: basic knowledges in bash.
  \end{itemize}

\end{minipage}







   

% \cventry{2009--2010}{Licence et Magistère de Mathématiques
% Fondamentales}{Faculté d'Orsay}{}{}{Mention bien, un module
% d'informatique théorique. Stage d'un mois dans le laboratoire LLR de
% l'école Polytechnique (Palaiseau): implémentation d'un filtre de
% Kalman.}  \cventry{2008--2009}{DEUG de Mathématiques}{Faculté
% d'Orsay}{}{}{Mention très bien.}

% \subsection{Productions écrites}
% \vspace{-0.4cm}
% \begin{center}
%   \begin{minipage}{0.7\linewidth}
%     Les documents présentés ici sont principalement des mémoires
%     réalisés durant mes études à l'université d'Orsay. Ils sont
%     décrits en détail et accessibles à l'adresse:
%     http://sebastien.besnier1.free.fr/
%   \end{minipage}
% \end{center}
% \cventry{2012}{Fonctions Elliptiques}{encadré par O. Bouillot}{}{}{}
% \cventry{2011}{Calcul Variationnel}{encadré par
% F. Santambrogio}{}{}{} \cventry{2011}{Algorithmique
% Quantique}{encadré par S. Laplante}{}{}{} \cventry{2009}{Articles de
% vulgarisation mathématique et informatique}{}{}{}{} \newpage




\section{Classical music}
\begin{itemize}
\item Guitar: since 1994
\item Piano: self-learning since 2000
\item Cello: since 2012
\item Lyric sing: irregularly since 2004
\end{itemize}
\end{document}